\section{はじめに}
\label{sec:start}

情報処理学会では,基幹論文誌として論文誌ジャーナルの発行を行っている.こ
れまで論文誌ジャーナル編集委員会では,論文誌ジャーナルの論文掲載時のフォー
マットとしてA4横型2段組を採用してきたが,会員からの多くの要望に基づき,
A4縦型2段組に変更することにした.また,これまでは投稿時と掲載時の形式が
異なっていたが,今回のフォーマット変更に合わせて,投稿時も掲載時と同様の
A4縦型2段組で受け付けることにした.


これに伴い,\LaTeX のスタイルファイルも新しいものに変更した.本稿では,
まずそのスタイルファイルを用いた論文のフォーマットに関して述べる.新たな
スタイルファイルでは,極力特別なコマンドは使わずに,標準的な \LaTeX のス
タイルを踏襲している.論文フォーマットに関しては,\ref{sec:format}~章で
後述する指針に従って頂くが,そこに規定されていること以外は標準的な\LaTeX
のコマンドをそのまま使うことができる.本稿は,そのスタイルファイルを実際
に使っているので,論文執筆の際に参考にされたい.


\footnotetext{本文は実際には論文誌ジャーナル編集委員会で作成したものを
EC2018実行委員が一部変更したものである.}

また,論文誌ジャーナル編集委員会では,論文の執筆する際に,著者がするべき
こと,するべきでないことを「べからず集」としてまとめた.本稿の後半に,論
文の内容に関する指針になるように,「べからず集」の内容をチェックリストと
してつけているので,投稿する前の内容のチェックに利用されたい.

%2
\section{EC2018への投稿の流れ}

%2.1
\subsection{準備}
EC2018の\LaTeX スタイルファイルを含む論文執筆キットは
%情報処理学会論文誌ジャーナルの \LaTeX スタイルファイルを含む論文執筆キッ
%トは
%\begin{quote}
%\small
%\|http://www.ipsj.or.jp/jip/submit/style.html|
%\end{quote}
EC2018Webサイトからダウンロードすることができる.
論文執筆キットは以下のファイルを含んでいる.
\begin{enumerate}
\item \|ec2018.cls     |: 原稿用クラスファイル
\item \|EC2018tech.sty |: 原稿用スタイルファイル
%\item \|ipsjdraft.sty |: 投稿用スタイル(査読用)
%\item \|ipsjpref.sty  |: 序文用スタイル
\item \|ec2018.tex     |: 本稿のソースファイル
%\item \|jsample.tex   |: 本稿のソースファイル
%\item \|esample.tex   |: 英文サンプルのソースファイル
\item \|ipsjsort.bst  |: jBibTEX スタイル(著者名順)
\item \|ipsjunsrt.bst |: jBibTEX スタイル(出現順)
\item \|bibsample.bib |: 文献リストのサンプル
%\item \|ebibsample.bib|: 英文文献リストのサンプル
\end{enumerate}
本キットはWindows (DOS)用として用意されている.また,実行環境としては
\LaTeXe を前提としているので,準備されたい.


%2.2
\subsection{最終原稿の作成と投稿}

エンタテインメントコンピューティングの
対象領域が広がった今,全ての発表者に等しく議論の場を提供するためには,
適切な発表形式の柔軟な選択が肝要である.
発表の中には理論構築が主であるために口頭発表が適切なものもあれば,
実体験による直感的理解が発表内容の理解を促進するためにデモ発表が
適切なものもある.

EC2010以降EC2016まで,口頭発表とデモ発表の原稿格差を撤廃するという理想のもと,
両発表とも原稿フォーマットおよび制限ページ数を同等とし,
違いがないものとして扱ってきた.

EC2017からは,デモも口頭発表も等しく扱うという考え方を継承しつつ,
デモにおいては「もの」を見せ,インタラクティブに議論することが一番重要であるという考え方を
さらに進め,「デモのみ」の発表区分を設けるとともに,原稿ページ数を2ページ以上とした.
一方で,口頭発表においてはロングとノートの2つを新たに設け,原稿ページ数をそれぞれ6〜10ページ,3〜5ページとした.
特に,口頭発表ノートについては萌芽的な研究を短時間で口頭発表しやすくするという配慮である.
さらにこれら口頭発表は,申し込みの際にデモも実施出来るように選択項目を設定してある.
論文フォーマットは本文章が書かれているこのLaTeXおよびWordフォーマット
を用い,原稿提出は\textbf{PDF形式}のファイルで行う.

EC2018にご投稿される執筆者各位は,全ての投稿をこの定型フォーマットを
利用して作成されたい.

EC2018では発表種別を3種類とし,それぞれに対応する発表時間と原稿のページ数を設定している( \tabref{tab:presentation}).

\begin{table}[tb]
\caption{発表種別と発表時間,原稿ページ数}
%\ecaption{Presentation type, time and the number of pages.}
\label{tab:presentation}
\hbox to\hsize{\hfil
\begin{tabular}{l|cr}\hline\hline
& 発表時間(質疑) & 原稿ページ数 \\\hline
口頭発表ロング &	18 min.(7 min.) & 6 -- 10 \\
口頭発表ノート &	10 min.(5 min.)       & 3 -- 5 \\
デモ &	--- & 2 --  \\\hline
\end{tabular}\hfil}
\end{table}



EC2018ではページ数の制限について,下限はデモのみ発表の2ページで,上限は
口頭発表ロングの10ページとした.また,
下限は,タイトルと概要のみだけでなく,それなりに発表内容が伝わる程度に
執筆してもらいたいという理由で設けている.

また,すべての原稿は\textbf{USB メモリにて電子的に}配布される.
したがって,図表などにカラー画像が利用できるが,読者が印刷して
利用する可能性もあることを考慮した上でのカラー画像利用を期待する.

本稿に従って用意した投稿用原稿の \LaTeX ソースからpdfファイルを作成し,
Adobeのpdf readerで読めることを確認した後,指定されたURLから投稿する.
EC2018 Webサイトを参照されたい.

%\subsection{著者校正・組版・出版}

%学会では用語や用字を一定の基準に従って修正することがある.また \LaTeX の
%実行環境の差異などによって著者が作成したハードコピーと実際の組版結果が微
%妙に異なることがある.これらの修正や差異が問題ないかを最終的に確認するた
%めに,著者にゲラ刷りが送られるので,もし問題があれば朱書によって指摘して
%返送する.なお{\bf この段階での記述誤りの修正は原則として認められない}の
%で,原稿送付時に細心の注意を払っていただきたい.

%その後,著者の校正に基づき最終的な組版を行ない,オンライン出版する.

%\newpage%%%!!!
