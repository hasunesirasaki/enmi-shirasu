\section{相互作用における味の提示実験}
% 目的
本実験では前述した手法を合わせることによる,リアルタイムに味を変化させる新たなデバイスを用いて,味覚に対して味を想起させることができるかを検討する.
スープの味には余計なものを感じさせない単純な塩味として塩とうま味成分を使用する.それに加え,香りをもたらすカップを使用することで味を想起させる.
今回の実験では,新たに作成した味覚変容デバイスの有用性を調査するとともに,新しく検証する塩味に対してアプローチがどの程度の認知を得るかを調査する実験である.

%3.4 全体の方法
\subsection{実験の方法}
まず実験を始める前に,直近で味のあるものをとったかどうかを確認し,目安30分以上開けて実験を遂行した.最初に,本実験では味の評価をしてもらう実験だと伝え,食べてもらうのは醤油と味噌の味であることを理解させた.

その後,評価用紙を提示し,それらを食べた経験があるかを問う.(この行程は考え中)この間に,実験者はスープの準備を行う.
スープはポットから用意するが,その作業風景は見せないようにした.
スープは・・・作り方,分量,中身の分量を記す・・・


そして,用意されたスープを作成したデバイスを用いて飲んでもらうように指示する.スープをのみ終わる,もしくは飲み終わりを宣言したら評価用紙に記入してもらい,その間にまた新しいスープを作成する.

この行程を2つの味+もとの味の3パターン行ってもらい,味の評価を4段階の評価(1:まったく感じない~4:すごく感じる)で変化の度合いを回答する.

%\begin{table}[tb]
%\caption{質問内容}
%\label{question}
%\hbox to\hsize{\hfil
%\begin{tabular}{l|lll}\hline\hline

%質問内容 & 評価 \\\hline
%香り:イチゴ(メロン)味 & 4段階評価\\\hline
%LED:イチゴ(メロン)味 & 4段階評価\\\hline
%香り+LED:イチゴ(メロン)味 & 4段階評価\\\hline
%\end{tabular}\hfil}
%\end{table}


なお,味の慣れや香りの混乱を避けるために一つの行程ごとに,被験者には評価用紙の記入してもらい,水で口を整えてもらう.実験者は容器を交換し新たなスープを用意することで時間を空けるようにした.
すべての味を食べた後には,食べてみたい味やコメントなどの自由意見を記入してもらうこととした.


本実験では,10代,20代の男女12名(男9名,女3名)に対しての調査を行った.

\begin{table}[tb]
\caption{質問内容}
\label{question}
\hbox to\hsize{\hfil
\begin{tabular}{l|lll}\hline\hline

質問内容 & 評価 \\\hline
醤油味だと感じましたか? & 4段階評価\\\hline
味噌味と感じましたか? & 4段階評価\\\hline
味だと感じましたか? & 4段階評価\\\hline
\end{tabular}\hfil}
\end{table}
