
\section{おわりに}


本研究では,LEDとエアポンプを用いたリアルタイムに複数の味を体験できる味覚変容デバイスを開発した.
そして視覚情報と嗅覚情報の重畳を利用した一瞬の味覚の変化を起こす実験を行った.
新しいデバイスは,食べやすさがかなり改善され,そのような指摘を受けることはなかった.
香りの配送装置に関しても,単純な作業で香りの切り替えをスムーズに行うことができた.以前の研究で使用したファンと今回のエアポンプでは,空気の出る量や,鼻に風が当たる時の表面積の違いがあるが,ユーザーに対して香りをしっかりと感じてもらうことができていた.また多少の香りは出ていたものの,部屋の香りの充満をおさえることができていた.


このシステムは必要な分だけの香りを提供するためにコンパクトに作られている.
それによって一つの食べ物に対して複数の場所から出すことができると考えている.
具体的には,綿菓子のような表面積の広いものに対してニオイを出す場所を細かく配置することで,この場所は何味、他の場所は違う味と言ったような楽しみ方も可能性としてはあり,今後の検討にしていきたい.

現段階においては,香りは2種類までしか出せず,それ以上のものを試すためにはチューブをはめなおしたりしなければいけない.その点をうまく切り替える方法を考えなくてはいけない.

また,提示する香りの強度についての確認が必要である.
自由記述の中には,「レモンの香りが強い」という意見が2割程度見られ,口頭でもそのように回答するユーザーが多かった.
一つの香りの強さが印象に残り,ほかの香りの印象が残らないことで,味に対して影響を与えることは十分に考えられる.そのため香りの強度の制御やその影響の調査を行っていく必要がある.

今後の展望としては,香りの強度の適切な量の特定や,他の香りを試していくことでの味の変化を調査していくことが挙げられる.
その中で,複数の香りを混ぜることでまた味に対して何かしらの違いを生むことができるか調査したいと考えている.また,食べている段階で色と香りを切り替えることで,その場でも味の変化を感じるのかを検証したいと考えている.

以上で述べたことを実践していくことで,馴染みのない香りや色の刺激による組み合わせや香りを与える順番を変えることでの残り香による味の影響といった,バリエーションのある新しい食体験を感じてもらえるのではないかと考えている.



%このシステムを用いて,かき氷に対して色と香りを重畳し,その見た目・香り・味を評価してもらう実験を行った.その結果,味の変化を感じているという結果を得られ,狙った通りの味を認識させることができるという有効性を示した.

%この手法は,かき氷に限らずさまざまな食べ物に対して一瞬の味を想起させることができるのではないかと考えている.具体的には,砂糖菓子のような単純な甘さがあり無臭のものでなら行えるのではないかと考えており,今後の検討にしていきたい.

%現段階において,複数の香りを試す際に,そのたびにデバイスのボディを開き中の脱脂綿を入れ替えなければいけない.その手間をなくすための香り伝達の仕組みを考えなくてはいけない.
%また,提示する香りに関しての換気の重要性についてを改めて再確認した.今回の実験では被験者が入れ替わるごとに,窓を開きファンを回して換気を行っていた.
%しかしながら,時間がたつにつれて徐々に香りが部屋に充満していってしまった.このことにより被験者が余計な香りに気づいてしまうことがあった.
%加えて,使用する脱脂綿にも香りが互いに移りあい,混ざり合った香りとなっていた.
%そのため適宜交換のタイミングを見極める必要がある.

%今後の展望として,食べている段階で色と香りを切り替えることで,その場でも味の変化を感じるのかを検証したいと考えている.
%そのために,色だけ変えたもの,香りだけを与えたもの,両方を与えたものを比較実験としてより多くの対象に実践していくことが必要である.そして換気を行いやすくするための部屋を作ることを検討している.

%以上で述べたことを実践していくことで,さまざまなバリエーションのある味として認識させることができるのではないかと考えている.また馴染みのない香りや色の刺激といった組み合わせを試すことで新しい体験となるのではないかと考えている.
