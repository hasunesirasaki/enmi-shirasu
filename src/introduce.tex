%1
\section{はじめに}
\label{sec:start}

人は食べ物のおいしさを味(味覚)だけでなく色や形(視覚),ニオイ(嗅覚),触感(触覚),食べ物の音(聴覚)など五感によってさまざまな刺激を外部から受けている.人の知覚というのはこれらの感覚が相互作用することにより形成されることが知られている.
本研究ではその中でも味覚に密接して関係のある視覚と嗅覚に着目してきた.

視覚は,食べ物が出てきたときに最初に働く感覚である.
食べる前からおいしそうだという感想を抱くことがあるように,見た目の色や形,潤いや光沢,盛り付け方と視覚から受ける料理の情報がおいしさを左右している.
そのため,料理の味付けだけではなく,見た目にこだわることでも味覚に変化を感じさせる重要なポイントとなる.
食材の切り方に工夫を入れること,多様な色の食材を使用して彩りを良くすること,料理の盛り付けを工夫することと言った見た目の重要性によって感じ方が変化する.

嗅覚は、「味」に大きな影響を与えている.
ニオイの分子は2つの嗅覚経路を通ることで嗅状皮細胞にたどり着く.一つは鼻から生じる経路で、一般的な嗅感覚である.もう一つは,口から鼻へと抜けていく経路で,何かものを食べたときに生じるものである.
嗅状皮細胞が特定の化学物質に触れることで人は香りを認識する.この感覚が脳で味覚と合成されることにより味が生まれる.
これは日々の体験からもよくわかる.嫌なものを食べるときには鼻をつまんでニオイを分からなくするようにするという経験は分かりやすい例である.
%このことからも人が味を認知するためにはまず鼻で匂いを感じ,それから舌で味を感知するという流れのもと,嗅覚の情報が手がかりとなっていることがわかる.


著者らはこれまでにおいて,味覚に対して一定の味の情報を与え,その他の情報を,視覚と嗅覚を用いて補うという形でさまざまな味を体験させる方法を模索し提案してきた.

これまでの研究\cite{fan}\cite{pomp}では「かき氷」を題材とし,味覚変容の手法を検討してきた.味の決め手となるシロップに対して,容器にLED光源を取り付けシロップの色を再現し,スプーンから香料を出すことで鼻に直接香りを与える.着色料や香料をシロップに混ぜ込むのではなく,別の情報として与えることで,一つの皿で様々な味をリアルタイムに切り替えるシステムを作成した.またその中で,そのシステムの有用性を示したと共に,味による様々な体験の違いを明らかにしてきた.
これらによって得られ結果は,2点ほどある.

まず1つ目に,かき氷のシロップにおける甘味において,シンプルな砂糖のシロップを下味として,視覚や嗅覚に別の味の情報を与えることで,砂糖をベースとした違う味をある程度知覚することができたということである.
2つ目に,違う味と知覚するうえでの情報量の割合は視覚よりも嗅覚が勝っているということが明らかとなった.
これらの知見を踏まえて,甘味に対してだけではなく他の五味に対しても同じように行うことができるのではないかと考え,塩味に対しても同じアプローチが可能ではないかと考えた.

塩味は,甘味と同様にプリミティブな味であり,様々な料理においてのおいしさのもとと言える.加えて,他の酸味や苦味に比べ,塩味をベースとしている料理がメジャーである.





本稿では,その知見を踏まえた上で,視覚と嗅覚によって味の知覚に表れる影響に差異があるのかを検討する.
