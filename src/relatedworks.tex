\section{関連研究}






\subsection{嗅覚と味覚}
嗅覚刺激による,知覚される味が変化することについての研究は古くからおこなわれており,それは単独の感覚ではなく,鼻から入る香りと口から入り鼻から抜けていく香りから成る2元性の感覚であると言われている\cite{keiro}.
普段の生活からあらわれることだが,風邪をひいたり花粉症になったりして鼻の調子がおかしくなった時,味覚は正常であっても感覚としておかしくなったと感じることがある.これは味覚の障害にとらわれることで自分の嗅覚に障害が生じたことを実感していない.書物の中には「味」の80パーセント以上は嗅覚に起因するものであると述べているものも少なくない\cite{book}.
このような例から嗅覚と味覚の間には相互作用が存在することが予想される.

嗅覚情報を提示するディスプレイは,さまざまに存在しており,それらに応じて様々な手法がとられている.

DavidらのinScent\cite{incent}は日常的な状況で着用できる嗅覚ディスプレイとしてネックレス型のウェアラブルデバイスを作成した.
これはSNSのメッセージ通知に対して嗅覚情報を追加することで連絡の認識を強めるためのものである.
ここでは,香りを発するシステムとしてアロマオイルを加熱によって気化させることで香りを生成している.
また香りの与える影響として,感情や記憶を呼び起こすファクターの一つであることを示した.

柳田らの局所的に香りを提示するための研究として,渦輪を利用して空気砲を送り,香りを搬送する香りプロジェクタ\cite{uzuwa}を開発し得られた知見をまとめている.
空気砲を使用することで狙った空間に香りを送り,また空気砲を互いにぶつけることで自然な流れの香りを届けることを試みている.
非装着と局所性という観点の技術的な難度と,突風間の少ない空間中の香り提示を実現する方法を見出した.

中村らは,嗅覚刺激によって方向提示を行う嗅覚デバイス\cite{nakamura1}を開発した.エアチューブを両側の鼻に当て,空気を送り込むことでどちらかの鼻に対して香りを送り込むシステムの制御を行っている.
香りの濃度の差異を知覚させることにより,方向の判断を下す材料としている.
香りの方向提示は可能であり,ある一つの香りが性別問わず高い確率で方向知覚が可能であることを明らかにしている.

これらの研究から目的によって,香りの配送は適した手法があることがよくわかる.
また嗅覚は感情や記憶を刺激し呼び起こすためのファクターとなっていることがわかる.
これは味の錯覚を起こさせるうえで重要なことの一つである.



\subsection{視覚と嗅覚における味覚}
近年では,バーチャルリアリティ(VR)技術の発展などにより,さまざまな感覚の情報が提示されるようになり,さまざまな行為や状況がバーチャルに体験できるようになっている.

鳴海らによるメタクッキー\cite{narumi2}は味覚に対して,嗅覚と視覚を外部刺激によって変化させる試みである.
プレーン味のクッキーに対してHMDを用いて見た目の違うクッキーに見せ,嗅覚的に別の味のクッキーの香りをエアポンプによる空気の送風で香りを嗅がせる.
視覚と嗅覚を用いることでの味の変化がある回答を得た.この方法で味の認識をねらい通りに得させることに対しての有用性を示した.

Ying-Liらの「TransFork」\cite{transfork}はVRによる視覚と,香りによる嗅覚を伴って味覚変換を体験するものである.
フォークに香りのついたボックスを取り付け,方向をユーザーの鼻に向けて調整することができ,ミニファンで臭いを誘導することができる.
食べ物の色を変えるために,QRコードを使ってフォークの位置を特定し,ヘッドマウントディスプレイを装着することによって増強された色を見ることができる.

この二つの研究はHMDを利用することで見た目を変化させているが,実際の食事ではそれを付けることはない.可用性に欠けたり,HMDをのせていることの違和感といったことは体験の妨げとなる可能性がある.そのため,現実的の視覚で行うことが良いと考える.

その一方でNimeshaらによるVocktail\cite{vocktail}は味覚,嗅覚,視覚を利用することで味の変化を与えるアプローチを行った.
視覚としてLEDによる色の印象,嗅覚として香り,味覚として電気味覚を使用し,水の風味がどのように変化をするかの実験を行った.
このシステムは3つの感覚の相互作用が仮想としての味覚に影響を与えていたことを明らかにしている.この手法は,無味である水から五味を表現しているが,食べ物の味の表現とするには無味から行うには難しい.


そのため本手法では,単純な味が食べ物の味を表現することがしやすいのではないかと仮定して,あらかじめに簡単な味をつけておき,そこに視覚・嗅覚を利用することで別の味を感じさせる手法をとった.
